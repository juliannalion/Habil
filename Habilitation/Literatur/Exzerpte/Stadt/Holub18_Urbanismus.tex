%Holub18_Urbanismus
\begin{itemize}
\item Definition urban entlehnt Lateinisch Urbanus \enquote{fein , Vornehmen, gebildet}
\enquote{Urbs} zur Stadt gehörend \autocite[136]{Holub18}.  
\item Sozialgeographie definiert das Urbane als funktionale Differenzierung, eine durch städtische Lebensweisen geprägte Alltagswelt und zielt auf Städtebauliche funktionale, sozio -- kulturelle und sozio -- ökonomische Elemente der Lebensumwelt ab
\autocite[136]{Holub18}.  
\item \textcite[136]{Holub18} definiert das urbane, oder besser gesagt Urbanität als \enquote{das Schaffen von offenen Orten der Begegnung, die einen sozialen Raum generieren.}
%\item Häußermann (2004) findet, das Städtische \enquote{sei die Koexistenz des Heterogenen auf engem Raum}. Diese muss mithilfe von Nutzungsbestimmungen, verschiedenen Gebäuden und einer Mischung unterschiedlichster kommerzielle Aktivitäten geplant werden. 
 \autocite[136]{Holub18}. 
% \item laut Häußermann ist Urbanität außerdem mit Öffentlichkeit verknüpft. Öffentlichkeit kann jedoch nur in unkontrollierten Räumen entstehen. Öffentlichkeit meint hierbei individuelles sowie kollektives Handeln, das sich aufeinander bezieht  \autocite[137]{Holub18}.
\item 
\autocite[14]{Holub18}. 
 \item 
\end{itemize}
